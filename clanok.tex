% Metódy inžinierskej práce

\documentclass[10pt,twoside,slovak,a4paper]{coursepaper}

\usepackage[slovak]{babel}
\usepackage[T1]{fontenc}
\usepackage[IL2]{fontenc} % lepšia sadzba písmena Ľ než v T1
\usepackage[utf8]{inputenc}
\usepackage{graphicx}
\usepackage{url} % príkaz \url na formátovanie URL
\usepackage{hyperref} % odkazy v texte budú aktívne (pri niektorých triedach dokumentov spôsobuje posun textu)

\usepackage{cite}
%\usepackage{times}

\pagestyle{headings}

\title{Evolúcia počítaťových hier\thanks{Semestrálny projekt v predmete Metódy inžinierskej práce, ak. rok 2022/23, vedenie: Ing. Ivan Kapustík}} % meno a priezvisko vyučujúceho na cvičeniach

\author{Dávid Babiš\\[2pt]
	{\small Slovenská technická univerzita v Bratislave}\\
	{\small Fakulta informatiky a informačných technológií}\\
	{\small \texttt{xbabis@stuba.sk}}
	}
	

\date{\small 17.10.2022} % upravte



\begin{document}

\maketitle

\begin{abstract}
To, že sa počítačové hry z roka na rok menia a vylepšujú, nie je pre vás určite novinkou. Vidieť ju však v rozmedzí mnoho rokov je zaručene niečo neopísateľné. Počítačová technika ide neuveriteľne rýchlo dopredu. Tieto hry sú s nami iba pár desiatok rokov, ale za tú dobu sa stihli výrazne zmeniť. Od jednoduchých pixelov na čiernobielych monitoroch sme sa dostali k prepracovanej grafike, ktorá nielen verne zachytáva našu skutočnosť, ale dokáže vykresliť aj neskutočnosť. Napríklad fantastické svety ktoré existujú len vo videohrách.  V tomto článku sa dozviete, ako sa menila nielen grafická stránka počítačových hier od roku 1952. Rozdiel je naozaj obrovský. Pamätáte si ešte niektoré staré klasiky?
\end{abstract}



\section{Úvod}
Ak sa chcete dozvedieť viac o vývine počítačových hier, zdokonaliť sa alebo ste o pc hrách ešte nič nevedeli, ste tu na správnom mieste. V tomto článku sa spoločne pozrieme na počítačové hry, ich históriu a rozdelenie podľa žánrov. Tiež si porovnáme starú a novú verziu jednej kultovej hry, a ukážeme si ako sa časom z hľadiska mechaniky a grafiky zmenila. Ešte sa pozrieme na evolúciu počítačových hier všeobecne, v priebehu desiatok rokov. V závere si tento pokrok ešte zhrnieme. História pc hier naznačená v úvode, je podrobnejšie vysvetlená v časti~\ref{historia}.
Dôležité súvislosti s rozdelením videohier podľa žánrov sú uvedené v časti~\ref{zanre}.
Záverečné poznámky prináša časť~\ref{zaver}.

\section{História počítačových hier} \label{historia}

\subsection{NIMROD 1951}

V roku 1951 bol pri príležitosti "Festival of Britain" predstavený digitálny počítač Ferranti NIMROD. Išlo o prvý počítač navrhnutý špeciálne pre počítačovú hru. Tento stroj so spotrebou krásnych 6 kilowattov a frekvenciou procesora 10kHz vedel, ako už názov napovedá, jedinú hru - NIM. Pôvodom pravdepodobne v Číne, NIM je jednoduchá logická hra pre dvoch hráčov spočívajúca v odoberaní prvkov z niekoľkých (typicky troch) množín. V každom ťahu môže hráč odobrať ľubovoľné množstvo prvkov (miniálne jeden) z jednej množiny. Víťazom je ten hráč, ktorý odoberie posledný prvok.

Ako „grafický“ výstup bol použitý panel so žiarovkami. NIMROD teda nie je mnohými považovaný za skutočnú videohru, pretože nepoužíva zobrazovacie zariadenie typu TV/monitor a pod. Ale to sa zase dostávame k problematickej definícii videohry. O niečo neskôr bol NIMROD s veľkým úspechom vystavený aj v Berlíne.



\subsection{OXO 1952} \label{nejaka}

Za prvú skutočnú videohru je možné OXO považovať najmä preto, že pre jej grafický výstup bola vôbec prvýkrát v dejinách počítačov použitá osciloskopická obrazovka, teda monitor.Vznikla v roku 1952.  Zobrazenie bitmapy bolo 35 × 16 pixelov. Hra bola nainštalovaná na elektrónkovom počítači EDSAC, ktorý bežne používal dierne pásky. Zahrať ste si mohli jedine proti umelej inteligencii. Ovládať ju bolo možné pomocou vytáčacieho telefónu, pričom vytočené číslo znamenalo políčko, na ktoré hráč umiestnil guľôčku alebo krížik.
Hra bola na počítači EDSAC nainštalovaná iba po dobu Douglesovej dizertačnej práce a širokej verejnosti sprístupnená nikdy nebola. Po obhájení svojej práce musel Douglles hru z počítača vymazať, pretože zaberala príliš mnoho miesta. 

Z obr.~\ref{f:rozhod} je všetko jasné. 

\begin{figure*}[tbh]
\centering
\includegraphics[scale=0.5]{uncharted4.jpg}
%Aj text môže byť prezentovaný ako obrázok. Stane sa z neho označný plávajúci objekt. Po vytvorení diagramu zrušte znak \texttt{\%} pred príkazom \verb|\includegraphics| označte tento riadok ako komentár (tiež pomocou znaku \texttt{\%}).
\caption{Rozhodujúci argument.}
\label{f:rozhod}
\end{figure*}



\section{Žánre počítačových hier} \label{zanre}

Základným problémom je teda že hry majú mnoho podôb a niekedy je ťažké ich rozlíšiť. Niektoré využívajú PvP\footnote{Player versus player - hráč proti hráčovi.}, iné zase AI\footnote{Artificial intelligence - umelá inteligencia.}.

Najprv sa pozrieme na nejaké vysvetlenie (časť~\ref{zanre:adventure}), a potom na ešte nejaké (časť~\ref{zanre:akcne}).\footnote{Niekedy môžete potrebovať aj poznámku pod čiarou.}

Môže sa zdať, že problém vlastne nejestvuje\cite{Coplien:MPD}, ale bolo dokázané, že to tak nie je~\cite{Czarnecki:Staged, Czarnecki:Progress}. Napriek tomu, aj dnes na webe narazíme na všelijaké pochybné názory\cite{PLP-Framework}. Dôležité veci možno \emph{zdôrazniť kurzívou}.


\subsection{Dobrodružné hry} \label{zanre:adventure}

Najskôr mali textovú podobu, no postupne sa vyvíjali až do dnešnej podoby, kde už ide o rozsiahle konverzácie postáv, rôzne zápletky, hádanky a bohatý príbeh. Ide väčšinou o single player hry, ale existujú aj takzvané co-op\footnote{Kooperatívne hry - skupina ľudí hrá hru, v ktorej hrajú na rovnakej strane, napríklad proti počítaču.}hry. Adventúry majú zvyčajne jednu hlavnú dejovú líniu a veľa vedlajších, tzv. side questov. Vyskytujú sa v nich aj rôzne easter eggy\footnote{Easter egg alebo veľkonočné vajíčko/vajce je skrytá a oficiálne nedokumentovaná funkcia alebo vlastnosť počítačového programu, DVD alebo CD. Veľkonočnými vajíčkami môžu byť neškodné minihry, vtipné funkcie, neobvyklé, implicitne nedefinované správanie programu, rôzne animácie, grafické symboly, titulky s menami tímu programových tvorcov.}. Hráč sa posúva v príbehu ďalej interakciou s prostredém, komunikáciou s rôznymi postavami, riešením hádaniek, zbieraním a používaním rôznych predmetov. Väčšinou ide o hry kde sa preskúmavajú hrobky, stratené mestá, hľadá sa poklad a podobne. Hráč sa môže nachádzať aj vo fiktívnom svete plného magických postáv, zvierat a predmetov. ale existujú aj realistické adventúrne hry. Adventúry nemajú na vyhranie kampane žiadny časpvý limit.

Príklady dobrodružných hier:

\begin{itemize}
\item Séria videohier Uncharted
\item Séria videohier Tomb Raider
\item The Last of Us
\item Days Gone
\item Death Stranding
\item Horizon Zero Dawn
\end{itemize}

\begin{figure*}[tbh]
\centering
\includegraphics[scale=0.5]{uncharted4.jpg}
\caption{Screenshot z dobrodružnej hry Uncharted 4}
\label{f:uncharted}
\end{figure*}

Ten istý zoznam, len číslovaný:

\begin{enumerate}
\item jedna vec
\item druhá vec
	\begin{enumerate}
	\item x
	\item y
	\end{enumerate}
\end{enumerate}


\subsection{Akčné hry} \label{zanre:akcne}

Patria sem rôzne bojové hry, strielačky ale aj takzvané FPS\footnote{First Person Shooter - hry z pohľadu prvej osoby} . Dôležitou vlastnosťou je mať dobré reflexy pretože hra sa odohráva v reálnom čase a hráč je pod nátlakom a má málo času na rozhodovanie. Hráči majú k dispozícii rôzne strelné a tiež ručné zbrane, granáty, nožíky a podobne. Akčné hry sú väčšinou časovo obmedzené. Tento žáner zvyčajne máva single player mód ale aj online mód.

\subsection{Battle Royale} \label{zanre:battleroyale}

Battle royale hry sú pomerne nový žáner. Sú veľmi podobné akčným hrám, s tým rozdielom že battle royale hry nemajú príbeh a ide hlavne o to, aby si hráč čo najviac užil a vychutnal akciu. Hlavným znakom je že hráči bojujú medzi sebou, každý proti každému. Hra sa odohráva na nejakom ostrove alebo na vymedzenej časti pevniny. Nachádza sa tu bezpečná zóna, ktorá je najskôr na celej mape, ale tá sa postupne zmenšuje a to núti hráčov k častejšiemu stretávaniu a konfliktom. Bezpečná zóna sa zmenšuje dovtedy, kým sú nažive ešte aspoň dvaja hráči. Ak hráč vystúpi z bezpečnej zóny postupne stráca životy a zomiera. Všetci hráči majú dostupnú mapu ostrova s vyznačenými lokalitami a bezpečnou zónou, prázdny inventár. Každý hráč má na začiatku hry rovanký počet životov a žiadne zbrane. Hra je väčšinou až pre 100 hráčov. Hráči začínajú hru niekde mimo ostrova, a po pripojení všetkých 100 hráčov, sa hráči presunú do lietadla, ktoré letí ponad mapou. Sami sa môžu rozdodnúť kde a kedy vyskočia z lietadla, a pomocou padáku pristanú na ľubovolnom mieste na mape. Po celej mape sú náhodne rozmiestnené zbrane a vozidlá. Cieľom hry je zbierať a vylepšovať rôzne zbrane a zvyšovače životov, liečiť sa, bojovať s ostatnými hráčmi, postupne sa presúvať po mape kvôli zmenšovaní bezpečnej zóny, prežiť a vyhrať. Hru vyhráva posledný živý hráč. Hra trvá približne 20-30 minút. Ide vždy o online hry.

\subsection{Strategické hry} \label{zanre:strategicke}

Sú určené pre hráčov, ktorých baví rôzne plánovanie útokov, vymýšlanie stratégií a rýchle konanie. Cielom je pomocou svojich bojových zručností, skúseností a dôvtipu vyhrať nad nepriateľom. Hráči svoje ťahy pečlivo plánujú a snažia sa prechytračiť oponenta. Svoju taktiku môžu meniť podľa vývoju hry. Stratégie sú vačšinou zasadené do stredoveku, kde bojujú pomocou vtedajších zbraní. Často má hráč na starosti napríklad nejakú pevnosť alebo kráľovstvo, a jeho úlohou je ju brániť, postupne zlepšovať obranu a hlavne útočiť na pevnosti ostatných hráčov. Tieto hry sú väčinou v online móde, ale môžu mať aj single player.  Typickou strategicou hrou je napríklad šach.

\subsection{RPG} \label{zanre:rpg}



\subsection{Simulátory} \label{zanre:simulatory}

V týchto hrách ide o simuláciu reálneho života, cielom je hlavne zabaviť sa, a to spôsobom robenia niečoho, čo v realite buď nemôžeme robiť alebo nevieme robiť a chceme sa to naučiť. Existuje mnoho simulátorov, najznámejšie sú Euro Truck Simulator, Farming Simulator, Flight Simulator a veľa ďalších. V skratke, existuje simulátor skoro na čokoľvek čo sa dá vykonávať v reálnom svete. Nejde o to hru vyhrať, a ani sa nedá vyhrať, ale hrať dovtedy, kým to hráča bude baviť.



\paragraph{Veľmi dôležitá poznámka.}
Niekedy je potrebné nadpisom označiť odsek. Text pokračuje hneď za nadpisom.



\section{Porovnanie CS 1.6 a CS:GO} \label{porovnanie}




\section{Ešte dôležitejšia časť} \label{dolezitejsia}




\section{Záver} \label{zaver} % prípadne iný variant názvu


%\input{example.tex}



%\acknowledgement{Ak niekomu chcete poďakovať\ldots}


% týmto sa generuje zoznam literatúry z obsahu súboru literatura.bib podľa toho, na čo sa v článku odkazujete
\bibliography{literatura}
\bibliographystyle{alpha} % prípadne alpha, abbrv alebo hociktorý iný
\end{document}
